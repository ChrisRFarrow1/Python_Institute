\documentclass[a4paper,10pt]{article}
\usepackage[utf8]{inputenc}
% The minted Package is for programming language typesetting
\usepackage{minted}
% The csquotes Package is to apply quotes to the document
\usepackage{csquotes}
% The tgbonum Package is for different font types
\usepackage{tgbonum}

%Opening Section
\title{Python Essentials - Module 2}
\author{Cisco Networking Academy}
%\date{}

\begin{document}
\maketitle{}
\tableofcontents
\addcontentsline{toc}{section}{Abstract}

%Abstract Section for Module 2
\begin{abstract}
In this module you'll learn:
\begin{itemize}
 \item how to write and run simple Python programs;
 \item what Python literals, operators, and expressions are;
 \item what variables are and what are the rules that govern them;
 \item how to perform basic input and output operations.
\end{itemize}
\end{abstract}

%Module 2, Section 1 - Your Very First Program
\section{Your Very First Program}
\subsection{Hello, World!}
It's time to start writing some real, working Python code. It'll be very simple for the time being.
\newline

As we're going to show you some fundamental concepts and terms, these snippets of code won't be serious or complex.
\newline

Run the code in the editor window on the right. If everything goes okay here, you'll see the line of text in the console window.
\newline

Alternatively, launch IDLE, create a new Python source file, fill it with this code, name the file and save it. Now run it. If everything goes okay, you'll see the rhyme's line in the IDLE console window. The code you have run should look familiar. You saw something very similar when we led you through the setting up of the IDLE environment.
\newline

Now we'll spend some time showing and explaining to you what you're actually seeing, and why it looks like this.
\newline

As you can see, the first program consists of the following parts:
\begin{itemize}
 \item the word
 \begin{minted}{python}
 print
 \end{minted}
 \item an opening parenthesis;
 \item a quotation mark;
 \item a line of text:
  \begin{minted}{python}
   Hello, World!
  \end{minted}
 \item another quotation mark;
 \item a closing parenthesis.
\end{itemize}

Each of the above plays a very important role in the code.

\begin{minted}{python}
print("Hello, World")
\end{minted}

\subsection{The print() Function}
Look at the line of code below:
\begin{minted}{python}
print("Hello, World")
\end{minted}
The word \textbf{print} that you can see here is a \textbf{function name}. That doesn't mean that wherever the word appears it is always a function name. The meaning of the word comes from the context in which the word has been used.
\newline

You've probably encountered the term function many times before, during math classes. You can probably also list several names of mathematical functions, like sine or log.
\newline

Python functions, however, are more flexible, and can contain more content than their mathematical siblings.
\newline

A function (in this context) is a separate part of the computer code able to:
\begin{itemize}
 \item \textbf{cause some effect} (e.g., send text to the terminal, create a file, draw an image, play a sound, etc.); this is something completely unheard of in the world of mathematics;
 \item \textbf{evaluate a value} (e.g., the square root of a value or the length of a given text) and \textbf{return it as the function's result}; this is what makes Python functions the relatives of mathematical concepts.
\end{itemize}

Moreover, many of Python functions can do the above two things together.
\newline

Where do the functions come from?
\begin{itemize}
 \item They may come from Python itself; the print function is one of this kind; such a function is an added value received together with Python and its environment (it is built-in); you don't have to do anything special (e.g., ask anyone for anything) if you want to make use of it;
 \item they may come from one or more of Python's add-ons named modules; some of the modules come with Python, others may require separate installation - whatever the case, they all need to be explicitly connected with your code (we'll show you how to do that soon);
 \item you can write them yourself, placing as many functions as you want and need inside your program to make it simpler, clearer and more elegant.
\end{itemize}

The name of the function should be significant (the name of the print function is self-evident).
\newline

Of course, if you're going to make use of any already existing function, you have no influence on its name, but when you start writing your own functions, you should consider carefully your choice of names.
\newline

As we said before, a function may have:
\begin{itemize}
 \item an \textbf{effect};
 \item a \textbf{result}.
\end{itemize}

There's also a third, very important, function component - \textbf{the argument(s)}.
\newline

Mathematical functions usually take one argument, e.g., sin(x) takes an x, which is the measure of an angle.
\newline

Python functions, on the other hand, are more versatile. Depending on the individual needs, they may accept any number of arguments - as many as necessary to perform their tasks. Note: any number includes zero - some Python functions don't need any argument.
\newline

In spite of the number of needed/provided arguments, Python functions strongly demand the presence of a pair of parentheses - opening and closing ones, respectively.
\newline

If you want to deliver one or more arguments to a function, you place them inside the parentheses. If you're going to use a function which doesn't take any argument, you still have to have the parentheses.
\newline

Note: to distinguish ordinary words from function names, place a pair of empty parentheses after their names, even if the corresponding function wants one or more arguments. This is a standard convention.
\newline

The function we're talking about here is {\fontfamily{cmtt}\selectfont print() }.
\newline

Does the {\fontfamily{cmtt}\selectfont print() }function in our example have any arguments?
\newline

Of course it does, but what are they?
\newline

The only argument delivered to the {\fontfamily{cmtt}\selectfont print()} function in this example is a string:
\begin{minted}{python}
 print("Hello, World!")
\end{minted}

As you can see, the \textbf{string is delimited with quotes} - in fact, the quotes make the string - they cut out a part of the code and assign a different meaning to it.
\newline

You can imagine that the quotes say something like: the text between us is not code. It isn't intended to be executed, and you should take it as is.
\newline

Almost anything you put inside the quotes will be taken literally, not as code, but as data. Try to play with this particular string - modify it, enter some new content, delete some of the existing content.
\newline

There's more than one way to specify a string inside Python's code, but for now, though, this one is enough.
\newline

So far, you have learned about two important parts of the code: \textbf{the function} and \textbf{the string}. We've talked about them in terms of syntax, but now it's time to discuss them in terms of semantics.
\newline

The function name (print in this case) along with the parentheses and argument(s), forms the function invocation.
\newline

We'll discuss this in more depth soon, but we should just shed a little light on it right now.
\begin{minted}{python}
 print("Hello, World!")
\end{minted}

What happens when Python encounters an invocation like this one below?
\newline

{\fontfamily{cmtt}\selectfont function\underline{\hspace{.10in}}name(argument)}
\newline

Let's see:
\begin{enumerate}
 \item Python checks if the name specified is legal (it browses its internal data in order to find an existing function of the name; if this search fails, Python aborts the code);
 \item Python checks if the function's requirements for the number of arguments allows you to invoke the function in this way (e.g., if a specific function demands exactly two arguments, any invocation delivering only one argument will be considered erroneous, and will abort the code's execution);
 \item Python leaves your code for a moment and jumps into the function you want to invoke; of course, it takes your argument(s) too and passes it/them to the function;
 \item The function executes its code, causes the desired effect (if any), evaluates the desired result(s) (if any) and finishes its task;
 \item Python returns to your code (to the place just after the invocation) and resumes its execution.
\end{enumerate}

Three important questions have to be answered as soon as possible:
\newline

\textbf{1. What is the effect the {\fontfamily{cmtt}\selectfont print()} function causes?}
\newline

The effect is very useful and very spectacular. The function:
\begin{itemize}
 \item takes its arguments (it may accept more than one argument and may also accept less than one argument)
 \item converts them into human-readable form if needed (as you may suspect, strings don't require this action, as the string is already readable)
 \item and sends the resulting data to the output device (usually the console); in other words, anything you put into the print() function will appear on your screen.
\end{itemize}

No wonder then, that from now on, you'll utilize {\fontfamily{cmtt}\selectfont print()} very intensively to see the results of your operations and evaluations.
\newline

\textbf{2. What arguments does {\fontfamily{cmtt}\selectfont print()} expect?}
\newline

Any. We'll show you soon that {\fontfamily{cmtt}\selectfont print()} is able to operate with virtually all types of data offered by Python. Strings, numbers, characters, logical values, objects - any of these may be successfully passed to {\fontfamily{cmtt}\selectfont print()}.
\newline

\textbf{3. What value does {\fontfamily{cmtt}\selectfont print()} function return?}
\newline

None. Its effect is enough.

\subsubsection{The print() function - instructions}
You have already seen a computer program that contains one function invocation. A function invocation is one of many possible kinds of Python instructions.
\newline

Of course, any complex program usually contains many more instructions than one. The question is: how do you couple more than one instruction into the Python code?
\newline

Python's syntax is quite specific in this area. Unlike most programming languages, \textbf{Python requires that there cannot be more than one instruction in a line}.
\newline

A line can be empty (i.e., it may contain no instruction at all) but it must not contain two, three or more instructions. This is strictly prohibited.
\newline

\textbf{Note}: Python makes one exception to this rule - it allows one instruction to spread across more than one line (which may be helpful when your code contains complex constructions).
\newline

Let's expand the code a bit, you can see it in the editor. Run it and note what you see in the console.
\newline

Your Python console should now look like this:
\newline

{\fontfamily{cmtt}\selectfont The itsy bitsy spider climbed up the waterspout.

Down came the rain and washed the spider out.}
\newline

This is a good opportunity to make some observations:
\begin{itemize}
 \item the program invokes the {\fontfamily{cmtt}\selectfont print()} \textbf{function twice}, and you can see two separate lines in the console - this means that {\fontfamily{cmtt}\selectfont print()} begins its output from a new line each time it starts its execution; you can change this behavior, but you can also use it to your advantage;
 \item each {\fontfamily{cmtt}\selectfont print()} invocation contains a different string, as its argument and the console content reflects it - this means \textbf{that the instructions in the code are executed in the same order} in which they have been placed in the source file; no next instruction is executed until the previous one is completed (there are some exceptions to this rule, but you can ignore them for now)
\end{itemize}


We've changed the example a bit - we've added one \textbf{empty} {\fontfamily{cmtt}\selectfont print()} function invocation. We call it empty because we haven't delivered any arguments to the function.
\newline

You can see it in the editor window. Run the code.
\newline

What happens?
\newline

If everything goes right, you should see something like this:
\newline

{\fontfamily{cmtt}\selectfont The itsy bitsy spider climbed up the waterspout.
\newline

Down came the rain and washed the spider out.}
\newline

As you can see, the empty {\fontfamily{cmtt}\selectfont print()} invocation is not as empty as you may have expected - it does output an empty line, or (this interpretation is also correct) its output is just a newline.
\newline

This is not the only way to produce a \textbf{newline} in the output console. We're now going to show you another way.

\subsubsection{The print() function - the escape and newline characters}
We've modified the code again. Look at it carefully.
\begin{minted}{python}
 print("The itsy bitsy spider\nclimbed up the waterspout.")
 print()
 print("Down came the rain\nand washed the spider out.")
\end{minted}

There are two very subtle changes - we've inserted a strange pair of characters inside the rhyme. They look like this: {\fontfamily{cmtt}\selectfont \textbackslash n}.
\newline

Interestingly, while \textbf{you can see two characters, Python sees one.}
\newline

The backslash ({\fontfamily{cmtt}\selectfont \textbackslash }) has a very special meaning when used inside strings - this is called \textbf{the escape character}.
\newline

The word escape should be understood specifically - it means that the series of characters in the string escapes for the moment (a very short moment) to introduce a special inclusion.
\newline

In other words, the backslash doesn't mean anything in itself, but is only a kind of announcement, that the next character after the backslash has a different meaning too.
\newline

The letter {\fontfamily{cmtt}\selectfont n} placed after the backshlash comes from the work \textit{newline}.
\newline

Both the backslash and the {\fontfamily{cmtt}\selectfont n} form a special symbol named \textbf{a newline character}, which urges the console to start a \textbf{new output line}.
\newline

Run the code. Your console should now look like this:
\newline

{\fontfamily{cmtt}\selectfont The itsy bitsy spider

climbed up the waterspout.
\newline

Down came the rain

and washed the spider out.}


As you can see, two newlines appear in the nursery rhyme, in the places where the {\fontfamily{cmtt}\selectfont \textbackslash n} have been used.
\newline

This convention has two important consequences:
\newline

1. If you want to put just one backslash inside a string, don't forget its escaping nature - you have to double it, e.g., such an invocation will cause an error:
\newline

{\fontfamily{cmtt}\selectfont print("\textbackslash")}
\newline

while this one won't:
\newline

{\fontfamily{cmtt}\selectfont print("\textbackslash\textbackslash")}
\newline


Not all escape pairs (the backslash coupled with another character) mean something.
\newline

Experiment with your code in the editor, run it, and see what happens.
\newline

\subsubsection{The print() function - using multiple arguments}
So far we have tested the {\fontfamily{cmtt}\selectfont print()} function behavior with no arguments, and with one argument. It's also worth trying to feed the {\fontfamily{cmtt}\selectfont print()} function with more than one argument.
\newline

This is what we're going to test now:
\begin{minted}{python}
 print("The itsy bitsy spider","climbed up","the water spout.")
\end{minted}


There is one {\fontfamily{cmtt}\selectfont print()} function invocation, but it contains \textbf{three arguments}. All of them are strings.
\newline

The arguments are \textbf{separated by commas}. We've surrounded them with spaces to make them more visible, but it's not really necessary, and we won't be doing it anymore.
\newline

In this case, the commas separating the arguments play a completely different role than the comma inside the string. The former is a part of Python's syntax, the latter is intended to be shown in the console.
\newline

If you look at the code again, you'll see that there are no spaces inside the strings.
\newline

Run the code and see what happens.
\newline

The console should now be showing the following text:


{\fontfamily{cmtt}\selectfont The itsy bitsy spider climbed up the waterspout.}
\newline

The spaces, removed from the strings, have appeared again. Can you explain why?
\newline

Two conclusions emerge from this example:
\begin{itemize}
 \item a {\fontfamily{cmtt}\selectfont print()} function invoked with more than one argument \textbf{outputs them all on one line};
 \item the {\fontfamily{cmtt}\selectfont print()} function \textbf{puts a space between the outputted arguments} on its own initiative.
\end{itemize}

\subsubsection{The print() function - the positional way of passing the arguments}
Now that you know a bit about {\fontfamily{cmtt}\selectfont print()} function customs, we're going to show you how to change them.
\newline

You should be able to predict the output without running the code in the editor.
\begin{minted}{python}
 print("My name is", "Python.")
 print("Monty Python.")
\end{minted}




Output will be:
\newline


{\fontfamily{cmtt}\selectfont My name is Python.


Monty Python.}
\newline



The way in which we are passing the arguments into the {\fontfamily{cmtt}\selectfont print()} function is the most common in Python, and is called \textbf{the positional way} (this name comes from the fact that the meaning of the argument is dictated by its position, e.g., the second argument will be outputted after the first, not the other way round).
\newline

\subsection{The print() function - the keyword arguments}
Python offers another mechanism for the passing of arguments, which can be helpful when you want to convince the {\fontfamily{cmtt}\selectfont print()} function to change its behavior a bit.
\newline

We aren't going to explain it in depth right now. We plan to do this when we talk about functions. For now, we simply want to show you how it works. Feel free to use it in your own programs.
\newline

The mechanism is called keyword arguments. The name stems from the fact that the meaning of these arguments is taken not from its location (position) but from the special word (keyword) used to identify them.
\newline

The {\fontfamily{cmtt}\selectfont print()} function has two keyword arguments that you can use for your purposes. The first of them is named {\fontfamily{cmtt}\selectfont end}.
\newline

You can see a very simple example of using a keyword argument:
\newline

\begin{minted}{python}
 print("My name is", "Python.", end=" ")
 print("Monty Python.")
\end{minted}



In order to use it, it is necessary to know some rules:
\begin{itemize}
 \item a keyword argument consists of three elements: a \textbf{keyword} identifying the argument ({\fontfamily{cmtt}\selectfont end} here); an \textbf{equal sign} ({\fontfamily{cmtt}\selectfont =}); and a \textbf{value} assigned to that argument;
 \item any keyword arguments have to be put \textbf{after the last positional argument} (this is very important)
\end{itemize}

In our example, we have made use of the {\fontfamily{cmtt}\selectfont end} keyword rgument, and set it to a string containing one space.
\newline

Run the code to see how it works.
\newline

The console should now be showing the following text:
\newline

{\fontfamily{cmtt}\selectfont My name is Python. Monty Python.}
\newline

As you can see, the {\fontfamily{cmtt}\selectfont end} keyword argument determines the characters the {\fontfamily{cmtt}\selectfont print()} function sends to the output once it reaches the end of its positional arguments.
\newline

The default behavior reflects the situation where the {\fontfamily{cmtt}\selectfont end} keyword argument is \textbf{implicitly} used in the following way: {\fontfamily{cmtt}\selectfont end="\textbackslash n"}. What this states is that instead of the default behavior of the end of that {\fontfamily{cmtt}\selectfont print()} function call - which is to hit the 'return' and move to the next line - it is replaced with simply a space (the space within the two double quotes). This is why {\fontfamily{cmtt}\selectfont Monty Python} is on the same line as {\fontfamily{cmtt}\selectfont My name is Python}, not the next line.
\newline

And now it's time to try something more difficult.


\begin{minted}{python}
 print("My name is ", end="")
 print("Monty Python.")
\end{minted}


If you look carefully, you'll see that we've used the {\fontfamily{cmtt}\selectfont end} argument, but the string assigned to it is empty (it contains no characters at all).
\newline

What will happen now? Run the program in the editor to find out.
\newline

As the {\fontfamily{cmtt}\selectfont end} argument has been set to nothing, the {\fontfamily{cmtt}\selectfont print()} function outputs nothing too, once its positional arguments have been exhausted.
\newline

The console should now be showing the following text:
\newline

{\fontfamily{cmtt}\selectfont My name is Monty Python.}
\newline

Note: \textbf{no newlines have been sent to the output.}
\newline

The string assigned to the end keyword argument can be of any length. Experiment with it if you want.
\newline



We've said previously that the {\fontfamily{cmtt}\selectfont print()} function separates its outputted arguments with spaces. This behavior can be changed, too.
\newline

The \textbf{keyword argument} that can do this is named {\fontfamily{cmtt}\selectfont sep} (like \textit{separator}). Here is the code:

\begin{minted}{python}
 print("My", "name", "is", "Monty", "Python.", sep="-")
\end{minted}



The {\fontfamily{cmtt}\selectfont sep} argument delivers the following results:
\newline

{\fontfamily{cmtt}\selectfont My-name-is-Monty-Python.}
\newline

What we gather is the default separator for the {\fontfamily{cmtt}\selectfont print()} function is a space, but in this instance the keyword argument {\fontfamily{cmtt}\selectfont sep} was used to print a '-' as the separator.
\newline

The {\fontfamily{cmtt}\selectfont print()} function now uses a dash, instead of a space, to separate the outputted arguments.
\newline

Note: the {\fontfamily{cmtt}\selectfont sep} argument's value may be an empty string, too. Try it for yourself.

\begin{minted}{python}
 print("My", "name", "is", "Monty", "Python.", sep="-")
\end{minted}


Output is:
{\fontfamily{cmtt}\selectfont MynameisMontyPython.}
\newline




Both keyword arguments \textbf{may be mixed in one invocation}, just like here in the editor window.
\newline

The example doesn't make much sense, but it visibly presents the interactions between {\fontfamily{cmtt}\selectfont end} and {\fontfamily{cmtt}\selectfont sep}.

\begin{minted}{python}
 print("My", "name", "is", sep="_", end="*")
 print("Monty", "Python.", sep="*", end="*\n")
\end{minted}


What would the output be?
\newline

{\fontfamily{cmtt}\selectfont My\underline{\hspace{.10in}}name\underline{\hspace{.10in}}is*Monty*Python*}
\newline

Now that you understand the {\fontfamily{cmtt}\selectfont print()} function, you're ready to consider how to store and process data in Python.
\newline

Without {\fontfamily{cmtt}\selectfont print()}, you wouldn't be able to see any results.

\subsection{LAB - The print() Function}
\subsubsection{Objectives}
\begin{itemize}
 \item becoming familiar with the {\fontfamily{cmtt}\selectfont print()} function and its formatting capabilities;
 \item experimenting with Python code.
\end{itemize}


\subsubsection{Scenario}
Modify the first line of code, using the {\fontfamily{cmtt}\selectfont sep} and {\fontfamily{cmtt}\selectfont end} keywords, to match the expected output. Use the two {\fontfamily{cmtt}\selectfont print()} functions in the editor.
\newline

Don't change anything in the second {\fontfamily{cmtt}\selectfont print()} invocation.
\newline

\subsubsection{Provided Code}
\begin{minted}{python}
 print("Programming","Essentials","in")
 print("Python")
\end{minted}


\subsubsection{Expected Code}
{\fontfamily{cmtt}\selectfont Programming***Essentials***in...Python}
\newline

\subsubsection{Modified Code}
\begin{minted}{python}
 print("Programming","Essentials","in",sep="***",end="...")
 print("Python")
\end{minted}

\subsection{LAB: Formatting the Output}
\subsubsection{Objectives}
\begin{itemize}
 \item experimenting with existing Python code;
 \item discovering and fixing basic syntax errors;
 \item becoming familiar with the {\fontfamily{cmtt}\selectfont print()} function and its formatting capabilities.
\end{itemize}

\subsubsection{Scenario}
We strongly encourage you to \textbf{play with the code} we've written for you, and make some (maybe even destructive) amendments. Feel free to modify any part of the code, but there is one condition - learn from your mistakes and draw your own conclusions.
\newline

Try to:
\begin{itemize}
 \item minimize the number of {\fontfamily{cmtt}\selectfont print()} function invocations by inserting the {\fontfamily{cmtt}\selectfont\textbackslash n} sequence into the strings
 \item make the arrow twice as large (but keep the proportions)
 \item duplicate the arrow, placing both arrows side by side; note: a string may be multiplied by using the following trick: {\fontfamily{cmtt}\selectfont "string" * 2} will produce {\fontfamily{cmtt}\selectfont "stringstring"} (we'll tell you more about it soon)
 \item remove any of the quotes, and look carefully at Python's response; pay attention to where Python sees an error - is this the place where the error really exists?
 \item do the same with some of the parentheses;
 \item change any of the {\fontfamily{cmtt}\selectfont print} words into something else, differing only in case (e.g., {\fontfamily{cmtt}\selectfont Print}) - what happens now?
 \item replace some of the quotes with apostrophes; watch what happens carefully.
\end{itemize}

\subsubsection{Provided Code}
\begin{minted}{python}
 print("    *")
 print("   * *")
 print("  *   *")
 print(" *     *")
 print("***   ***")
 print("  *   *")
 print("  *   *")
 print("  *****")
\end{minted}

\subsubsection{Modified Code}
\textbf{Challenge 1}
Minimize the number of {\fontfamily{cmtt}\selectfont print()} function invocations by inserting the {\fontfamily{cmtt}\selectfont\textbackslash n} sequence into the strings.
\newline

\begin{minted}{python}
 print("    *","   * *","  *   *","***   ***","  *   *","  *   *","  *****",sep="\n")
\end{minted}

\textbf{Challenge 2}
Duplicate the arrow, placing both arrows side by side; note: a string may be multiplied by using the following trick: {\fontfamily{cmtt}\selectfont "string" * 2} will produce {\fontfamily{cmtt}\selectfont "stringstring"}.
\newline

\begin{minted}{python}
 print("    *" *2)
 print("   * *" *2)
 print("  *   *" *2)
 print(" *     *" *2)
 print("***   ***" *2)
 print("  *   *" *2)
 print("  *   *" *2)
 print("  *****" *2)
\end{minted}

\subsection{Section Summary - Key Takeaways}
\begin{enumerate}
 \item The {\fontfamily{cmtt}\selectfont print()} function is a \textbf{built-in} function. It prints/outputs a specified message to the screen/consol window.
 \item Built-in functions, contrary to user-defined functions, are always available and don't have to be imported. Python 3.8 comes with 69 built-in functions. You can find their full list provided in alphabetical order in the Python Standard Library.
 \item To call a function (this process is known as \textbf{function invocation} or \textbf{function call}), you need to use the function name followed by parentheses. You can pass arguments into a function by placing them inside the parentheses. You must separate arguments with a comma, e.g., {\fontfamily{cmtt}\selectfont print("Hello,", "world!")}. An "empty" {\fontfamily{cmtt}\selectfont print()} function outputs an empty line to the screen.
 \item Python strings are delimited with \textbf{quotes}, e.g., {\fontfamily{cmtt}\selectfont "I am a string"} (double quotes), or {\fontfamily{cmtt}\selectfont 'I am a string, too'} (single quotes).
 \item Computer programs are collections of \textbf{instructions}. An instruction is a command to perform a specific task when executed, e.g., to print a certain message to the screen.
  \item In Python strings the \textbf{backslash} (\textbackslash) is a special character which announces that the next character has a different meaning, e.g., \textbackslash n (the \textbf{newline character}) starts a new output line.
 \item \textbf{Positional arguments} are the ones whose meaning is dictated by their position, e.g., the second argument is outputted after the first, the third is outputted after the second, etc.
 \item \textbf{Keyword arguments} are the ones whose meaning is not dictated by their location, but by a special word (keyword) used to identify them.
 \item The {\fontfamily{cmtt}\selectfont end} and {\fontfamily{cmtt}\selectfont sep} parameters can be used for formatting the output of the {\fontfamily{cmtt}\selectfont print()} function. The sep parameter specifies the separator between the outputted arguments (e.g., {\fontfamily{cmtt}\selectfont print("H", "E", "L", "L", "O", sep="-")}, whereas the {\fontfamily{cmtt}\selectfont end} parameter specifies what to print at the end of the print statement.
\end{enumerate}




%Module 2, Section 2 - Python Literals
\section{Python Literals}
\subsection{Literals - the Data in Itself}
Now that you have a little knowledge of some of the powerful features offered by the {\fontfamily{cmtt}\selectfont print()} unction, it's time to learn about some new issues, and one important new term - the \textbf{literal}.











\end{document}
