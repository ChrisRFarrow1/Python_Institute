\documentclass[a4paper,10pt]{article}
\usepackage[utf8]{inputenc}
% The minted Package is for programming language typesetting
\usepackage{minted}
% The csquotes Package is to apply quotes to the document
\usepackage{csquotes}
% The tgbonum Package is for different font types
%\usepackage{tgbonum}

%opening
\title{Python Essentials - Module 1}
\author{Cisco Networking Academy}

\begin{document}
\maketitle{}
\begin{abstract}
For module 1 the learning items are:
\begin{itemize}
    \item the fundamentals of computer programming, i.e., how the computer works, how the program is executed, how the programming language is defined and constructed;
    \item the difference between compilation and interpretation
    \item what Python is, how it is positioned among other programming languages, and what distinguishes the different versions of Python.
\end{itemize}
\end{abstract}

\section{How Does a Computer Work}
This course aims to show you what the Python language is and what it is used for. Let's start from the absolute basics.
\newline

A program makes a computer usable. Without a program, a computer, even the most powerful one, is nothing more than an object. Similarly, without a player, a piano is nothing more than a wooden box.
\newline

Computers are able to perform very complex tasks, but this ability is not innate. A computer's nature is quite different.
\newline

It can execute only extremely simple operations. For example, a computer cannot understand the value of a complicated mathematical function by itself, although this isn't beyond the realms of possibility in the near future.
\newline

Contemporary computers can only evaluate the results of very fundamental operations, like adding or dividing, but they can do it very fast, and can repeat these actions virtually any number of times.
\newline

Imagine that you want to know the average speed you've reached during a long journey. You know the distance, you know the time, you need the speed.
\newline

Naturally, the computer will be able to compute this, but the computer is not aware of such things as distance, speed, or time. Therefore, it is necessary to instruct the computer to:
\begin{itemize}
    \item accept a number representing the distance;
    \item accept a number representing the travel time;
    \item divide the former value by the latter and store the result in the memory;
    \item display the result (representing the average speed) in a readable format.
\end{itemize}

These four simple actions form a \textbf{program} Of course, these examples are not formalized, and they are very far from what the computer can understand, but they are good enough to be translated into a language the computer can accept.
\newline

\textbf{Language} is the keyword.

\section{Natural languages vs. programming languages}
A language is a means (and a tool) for expressing and recording thoughts. There are many languages all around us. Some of them require neither speaking nor writing, such as body language; it's possible to express your deepest feelings very precisely without saying a word.
\newline

Another language you use each day is your mother tongue, which you use to manifest your will and to ponder reality. Computers have their own language, too, called \textbf{machine language}, which is very rudimentary.
\newline

A computer, even the most technically sophisticated, is devoid of even a trace of intelligence. You could say that it is like a well-trained dog - it responds only to a predetermined set of known commands.
\newline

The commands it recognizes are very simple. We can imagine that the computer responds to orders like "take that number, divide by another and save the result".
\newline

A complete set of known commands is called an \textbf{instruction list}, sometimes abbreviated to \textbf{IL}. Different types of computers may vary depending on the size of their ILs, and the instructions could be completely different in different models.
\newline

Note: machine languages are developed by humans.
\newline

No computer is currently capable of creating a new language. However, that may change soon. Just as people use a number of very different languages, machines have many different languages, too. The difference, though, is that human languages developed naturally.
\newline

Moreover, they are still evolving, and new words are created every day as old words disappear. These languages are called \textbf{natural languages}.

\section{What Makes a Language}
We can say that each language (machine or natural, it doesn't matter) consists of the following elements:

\begin{itemize}
 \item an \textbf{alphabet}: a set of symbols used to build words of a certain language (e.g., the Latin alphabet for English, the Cyrillic alphabet for Russian, Kanji for Japanese, and so on)
  \item a \textbf{lexis}: (aka a dictionary) a set of words the language offers its users (e.g., the word "computer" comes from the English language dictionary, while "cmoptrue" doesn't; the word "chat" is present both in English and French dictionaries, but their meanings are different)
  \item a \textbf{syntax}: a set of rules (formal or informal, written or felt intuitively) used to determine if a certain string of words forms a valid sentence (e.g., "I am a python" is a syntactically correct phrase, while "I a python am" isn't)
  \item \textbf{semantics}: a set of rules determining if a certain phrase makes sense (e.g., "I ate a doughnut" makes sense, but "A doughnut ate me" doesn't)
\end{itemize}

The IL is, in fact, the \textbf{alphabet of a machine language}. This is the simplest and most primary set of symbols we can use to give commands to a computer. It's the computer's mother tongue.
\newline

Unfortunately, this tongue is a far cry from a human mother tongue. We all (both computers and humans) need something else, a common language for computers and humans, or a bridge between the two different worlds.
\newline

We need a language in which humans can write their programs and a language that computers may use to execute the programs, one that is far more complex than machine language and yet far simpler than natural language.
\newline

Such languages are often called high-level programming languages. They are at least somewhat similar to natural ones in that they use symbols, words and conventions readable to humans. These languages enable humans to express commands to computers that are much more complex than those offered by ILs.
\newline

A program written in a high-level programming language is called a source code (in contrast to the machine code executed by computers). Similarly, the file containing the \textbf{source code} is called the \textbf{source file}.

\section{Compilation versus Interpretation}
Computer programming is the act of composing the selected programming language's elements in the order that will cause the desired effect. The effect could be different in every specific case - it's up to the programmer's imagination, knowledge and experience.
\newline

Of course, such a composition has to be correct in many senses:
\begin{itemize}
 \item \textbf{alphabetically} - a program needs to be written in a recognizable script, such as Roman, Cyrillic, etc.
 \item \textbf{lexically} - each programming language has its dictionary and you need to master it; thankfully, it's much simpler and smaller than the dictionary of any natural language;
 \item \textbf{syntactically} - each language has its rules and they must be obeyed;
 \item \textbf{semantically} - the program has to make sense.
\end{itemize}

Unfortunately, a programmer can also make mistakes with each of the above four senses. Each of them can cause the program to become completely useless.
\newline

Let's assume that you've successfully written a program. How do we persuade the computer to execute it? You have to render your program into machine language. Luckily, the translation can be done by a computer itself, making the whole process fast and efficient.
\newline

There are two different ways of \textbf{transforming a program from a high-level programming language into machine language}:
\begin{itemize}
    \item The source program is translated once (however, this act must be repeated each time you modify the source code) by getting a file (e.g., an .exe file if the code is intended to be run under MS Windows) containing the machine code; now you can distribute the file worldwide; the program that performs this translation is called a compiler or translator.
    \item You (or any user of the code) can translate the source program each time it has to be run; the program performing this kind of transformation is called an interpreter, as it interprets the code every time it is intended to be executed; it also means that you cannot just distribute the source code as-is, because the end-user also needs the interpreter to execute it.
\end{itemize}

Due to some very fundamental reasons, a particular high-level programming language is designed to fall into one of these two categories.
\newline

There are very few languages that can be both compiled and interpreted. Usually, a programming language is projected with this factor in its constructors' minds - will it be compiled or interpreted?

\section{What does the interpreter actually do?}
Let's assume once more that you have written a program. Now, it exists as a \textbf{computer file}: a computer program is actually a piece of text, so the source code is usually placed in \textbf{text files}.
\newline

Note: it has to be \textbf{pure text}, without any decorations like different fonts, colors, embedded images or other media. Now you have to invoke the interpreter and let it read your source file.
\newline

The interpreter reads the source code in a way that is common in Western culture: from top to bottom and from left to right. There are some exceptions - they'll be covered later in the course.
\newline

First of all, the interpreter checks if all subsequent lines are correct (using the four aspects covered earlier).
\newline

If the interpreter finds an error, it finishes its work immediately. The only result in this case is an \textbf{error message}.
\newline

The interpreter will inform you where the error is located and what caused it. However, these messages may be misleading, as the interpreter isn't able to follow your exact intentions, and may detect errors at some distance from their real causes.
\newline

For example, if you try to use an entity of an unknown name, it will cause an error, but the error will be discovered in the place where it tries to use the entity, not where the new entity's name was introduced.
\newline

In other words, the actual reason is usually located a little earlier in the code, for example, in the place where you had to inform the interpreter that you were going to use the entity of the name.
\newline

If the line looks good, the interpreter tries to execute it (note: each line is usually executed separately, so the trio "read-check-execute" can be repeated many times - more times than the actual number of lines in the source file, as some parts of the code may be executed more than once).
\newline

It is also possible that a significant part of the code may be executed successfully before the interpreter finds an error. This is normal behavior in this execution model.
\newline

You may ask now: which is better? The "compiling" model or the "interpreting" model? There is no obvious answer. If there had been, one of these models would have ceased to exist a long time ago. Both of them have their advantages and their disadvantages.
\newline

\section{Compilation vs. Interpretation - Advantages and Disadvantages}
\subsection{Compilation}
\subsubsection{Advantages}
\begin{itemize}
 \item the execution of the translated code is usually faster;
 \item only the user has to have the compiler - the end-user may use the code without it;
 \item the translated code is stored using machine language - as it is very hard to understand it, your own inventions and programming tricks are likely to remain your secret.
\end{itemize}
\subsubsection{Disadvantages}
\begin{itemize}
 \item the compilation itself may be a very time-consuming process - you may not be able to run your code immediately after any amendment;
 \item you have to have as many compilers as hardware platforms you want your code to be run on.
\end{itemize}

\subsection{Interpretation}
\subsubsection{Advantages}
\begin{itemize}
 \item you can run the code as soon as you complete it - there are no additional phases of translation;
 \item the code is stored using programming language, not the machine one - this means that it can be run on computers using different machine languages; you don't compile your code separately for each different architecture.
\end{itemize}
\subsubsection{Disadvantages}
\begin{itemize}
 \item don't expect that interpretation will ramp your code to high speed - your code will share the computer's power with the interpreter, so it can't be really fast;
 \item both you and the end user have to have the interpreter to run your code.
\end{itemize}





\subsection{What does this all mean for you?}
\begin{itemize}
 \item Python is an \textbf{interpreted language}. This means that it inherits all the described advantages and disadvantages. Of course, it adds some of its unique features to both sets.
 \item If you want to program in Python, you'll need the \textbf{Python interpreter}. You won't be able to run your code without it. Fortunately, Python is free. This is one of its most important advantages.
\end{itemize}

Due to historical reasons, languages designed to be utilized in the interpretation manner are often called \textbf{scripting languages}, while the source programs encoded using them are called \textbf{scripts}.


\section{So What is Python?}
Python is a widely-used, interpreted, object-oriented, and high-level programming language with dynamic semantics, used for general-purpose programming.
\newline

And while you may know the python as a large snake, the name of the Python programming language comes from an old BBC television comedy sketch series called Monty Python's Flying Circus.
\newline

At the height of its success, the Monty Python team were performing their sketches to live audiences across the world, including at the Hollywood Bowl.
\newline

Since Monty Python is considered one of the two fundamental nutrients to a programmer (the other being pizza), Python's creator named the language in honor of the TV show.

\subsection{Who Created Python?}
One of the amazing features of Python is the fact that it is actually one person's work. Usually, new programming languages are developed and published by large companies employing lots of professionals, and due to copyright rules, it is very hard to name any of the people involved in the project. Python is an exception.
\newline

There are not many languages whose authors are known by name. Python was created by Guido van Rossum, born in 1956 in Haarlem, the Netherlands. Of course, Guido van Rossum did not develop and evolve all the Python components himself.
\newline

The speed with which Python has spread around the world is a result of the continuous work of thousands (very often anonymous) programmers, testers, users (many of them aren't IT specialists) and enthusiasts, but it must be said that the very first idea (the seed from which Python sprouted) came to one head - Guido's.

\subsection{A hobby programming project}
The circumstances in which Python was created are a bit puzzling. According to Guido van Rossum:
\begin{displayquote}
 In December 1989, I was looking for a "hobby" programming project that would keep me occupied during the week around Christmas. My office (...) would be closed, but I had a home computer, and not much else on my hands. I decided to write an interpreter for the new scripting language I had been thinking about lately: a descendant of ABC that would appeal to Unix/C hackers. I chose Python as a working title for the project, being in a slightly irreverent mood (and a big fan of Monty Python's Flying Circus).

 - Guido van Rossum
\end{displayquote}

\subsection{Python Goals}
In 1999, Guido van Rossum defined his goals for Python:
\begin{itemize}
 \item an easy and intuitive language just as powerful as those of the major competitors;
 \item open source, so anyone can contribute to its development;
 \item code that is as understandable as plain English;
 \item suitable for everyday tasks, allowing for short development times.
\end{itemize}
About 20 years later, it is clear that all these intentions have been fulfilled. Some sources say that Python is the most popular programming language in the world, while others claim it's the second or the third.

Python isn't a young language anymore. It is \textbf{mature and trustworthy}. It's not a one-hit wonder. It's a bright star in the programming firmament, and time spent learning Python is a very good investment.

\subsection{What Makes Python Special?}
How does it happen that programmers, young and old, experienced and novice, want to use it? How did it happen that large companies adopted Python and implemented their flagship products using it?

There are many reasons - we've listed some of them already, but let's enumerate them again in a more practical manner:
\begin{itemize}
 \item it's easy to learn - the time needed to learn Python is shorter than for many other languages; this means that it's possible to start the actual programming faster;
 \item it's easy to teach - the teaching workload is smaller than that needed by other languages; this means that the teacher can put more emphasis on general (language-independent) programming techniques, not wasting energy on exotic tricks, strange exceptions and incomprehensible rules;
 \item it's easy to use for writing new software - it's often possible to write code faster when using Python;
 \item it's easy to understand - it's also often easier to understand someone else's code faster if it is written in Python;
 \item it's easy to obtain, install and deploy - Python is free, open and multiplatform; not all languages can boast that.
\end{itemize}


Of course, Python has its drawbacks, too:
\begin{itemize}
 \item it's not a speed demon - Python does not deliver exceptional performance;
 \item in some cases it may be resistant to some simpler testing techniques - this may mean that debugging Python's code can be more difficult than with other languages; fortunately, making mistakes is always harder in Python.
\end{itemize}

It should also be stated that Python is not the only solution of its kind available on the IT market.
\newline

It has lots of followers, but there are many who prefer other languages and don't even consider Python for their projects.

\subsection{Python Rivals?}
Python has two direct competitors, with comparable properties and predispositions. These are:
\begin{itemize}
 \item \textbf{Perl} - a scripting language originally authored by Larry Wall;
 \item \textbf{Ruby} - a scripting language originally authored by Yukihiro Matsumoto.
The former is more traditional, more conservative than Python, and resembles some of the good old languages derived from the classic C programming language.
\end{itemize}

In contrast, the latter is more innovative and more full of fresh ideas than Python. Python itself lies somewhere between these two creations.
\newline

The Internet is full of forums with infinite discussions on the superiority of one of these three over the others, should you wish to learn more about each of them.

\subsection{Where can we see Python in action?}
We see it every day and almost everywhere. It's used extensively to implement complex \textbf{Internet services} like search engines, cloud storage and tools, social media and so on. Whenever you use any of these services, you are actually very close to Python, although you wouldn't know it.
\newline

Many \textbf{developing tools} are implemented in Python. More and more \textbf{everyday use applications} are being written in Python. Lots of scientists have abandoned expensive proprietary tools and switched to Python. Lots of IT project testers have started using Python to carry out repeatable test procedures. The list is long.

\subsection{Why not Python?}
Despite Python's growing popularity, there are still some niches where Python is absent, or is rarely seen:
\begin{itemize}
 \item low-level programming (sometimes called "close to metal" programming): if you want to implement an extremely effective driver or graphical engine, you wouldn't use Python;
 \item applications for mobile devices: although this territory is still waiting to be conquered by Python, it will most likely happen someday.
\end{itemize}

\subsection{There is more than one Python}
There are two main kinds of Python, called Python 2 and Python 3.
\newline

Python 2 is an older version of the original Python. Its development has since been intentionally stalled, although that doesn't mean that there are no updates to it. On the contrary, the updates are issued on a regular basis, but they are not intended to modify the language in any significant way. They rather fix any freshly discovered bugs and security holes. Python 2's development path has reached a dead end already, but Python 2 itself is still very much alive.
\newline

\textbf{Python 3 is the newer (to be precise, the current) version of the language. It's going through its own evolution path, creating its own standards and habits.}
\newline

These two versions of Python aren't compatible with each other. Python 2 scripts won't run in a Python 3 environment and vice versa, so if you want the old Python 2 code to be run by a Python 3 interpreter, the only possible solution is to rewrite it, not from scratch, of course, as large parts of the code may remain untouched, but you do have to revise all the code to find all possible incompatibilities. Unfortunately, this process cannot be fully automatized.
\newline

Python 3 isn't just a better version of Python 2 - it is a completely different language, although it's very similar to its predecessor. When you look at them from a distance, they appear to be the same, but when you look closely, though, you notice a lot of differences.
\newline

If you're modifying an old existing Python solution, then it's highly likely that it was coded in Python 2. This is the reason why Python 2 is still in use. There are too many existing Python 2 applications to discard it altogether.
\newline

If you're going to start a new Python project, you should use Python 3, and this is the version of Python that will be used during this course.
\newline

It is important to remember that there may be smaller or bigger differences between subsequent Python 3 releases (e.g., Python 3.6 introduced ordered dictionary keys by default under the CPython implementation) - the good news, though, is that all the newer versions of Python 3 are backwards compatible with the previous versions of Python 3. Whenever meaningful and important, we will always try to highlight those differences in the course.
\newline

All the code samples you will find during the course have been tested against Python 3.4, Python 3.6, Python 3.7, and Python 3.8.

\subsection{Python aka CPython}
In addition to Python 2 and Python 3, there is more than one version of each.
\newline

First of all, there are the Pythons which are maintained by the people gathered around the PSF (Python Software Foundation), a community that aims to develop, improve, expand, and popularize Python and its environment. The PSF's president is Guido von Rossum himself, and for this reason, these Pythons are called canonical. They are also considered to be \textbf{reference Pythons}, as any other implementation of the language should follow all standards established by the PSF.
\newline

Guido van Rossum used the "C" programming language to implement the very first version of his language and this decision is still in force. All Pythons coming from the PSF are written in the "C" language. There are many reasons for this approach and it has many consequences. One of them (probably the most important) is that thanks to it, Python may be easily ported and migrated to all platforms with the ability to compile and run "C" language programs (virtually all platforms have this feature, which opens up many expansion opportunities for Python).
\newline

This is why the PSF implementation is often referred to as CPython. This is the most influential Python among all the Pythons in the world.
\newline

\subsection{Cython}
Another Python family member is \textbf{Cython}.
\newline

Cython is one of a possible number of solutions to the most painful of Python's trait - the lack of efficiency. Large and complex mathematical calculations may be easily coded in Python (much easier than in "C" or any other traditional language), but the resulting code's execution may be extremely time-consuming.
\newline

How are these two contradictions reconciled? One solution is to write your mathematical ideas using Python, and when you're absolutely sure that your code is correct and produces valid results, you can translate it into "C". Certainly, "C" will run much faster than pure Python.
\newline

This is what Cython is intended to do - to automatically translate the Python code (clean and clear, but not too swift) into "C" code (complicated and talkative, but agile).

\subsection{Jython}
Another version of Python is called \textbf{Jython}.
\newline

"J" is for "Java". Imagine a Python written in Java instead of C. This is useful, for example, if you develop large and complex systems written entirely in Java and want to add some Python flexibility to them. The traditional CPython may be difficult to integrate into such an environment, as C and Java live in completely different worlds and don't share many common ideas.
\newline

Jython can communicate with existing Java infrastructure more effectively. This is why some projects find it usable and needful.
\newline

Note: the current Jython implementation follows Python 2 standards. There is no Jython conforming to Python 3, so far.

\subsection{PyPy and RPython}
\textbf{PyPy} - a Python within a Python. In other words, it represents a Python environment written in Python-like language named RPython (Restricted Python). It is actually a subset of Python.
\newline

The source code of PyPy is not run in the interpretation manner, but is instead translated into the C programming language and then executed separately.
\newline

This is useful because if you want to test any new feature that may be (but doesn't have to be) introduced into mainstream Python implementation, it's easier to check it with PyPy than with CPython. This is why PyPy is rather a tool for people developing Python than for the rest of the users.
\newline

This doesn't make PyPy any less important or less serious than CPython, of course.
\newline

In addition, PyPy is compatible with the Python 3 language.
\newline

There are many more different Pythons in the world. You'll find them if you look, but this course will focus on CPython.

\end{document}
